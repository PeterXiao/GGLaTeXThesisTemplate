\chapter{Konzeption}

Beantwortet werden müssen folgende Fragen:
\begin{itemize}
    \item Kernaufgaben?
    \item Nutzung wie und von wem? Ein Softwareprojekt betrifft in der Regel viele unterschiedliche Interessensgruppen, die sogenannten Stakeholder des Projekts.
    Seien es die involvierten Abteilungen, der Projektsponsor oder die späteren Benutzer: alle Stakeholder haben (manchmal widersprechende) Anforderungen an die zu entwickelnde Software.
    \item Schnittstellen?
    \item Welche Daten?
    \item Darstellung?
    \item Steuerung?
    \item Eigenschaften?
\end{itemize}

\section{Fachliche Analyse}

WAS soll die Software tun?

\section{Technische Analyse}

 WIE soll die Software das tun? (technische Umsetzung)

\section{Funktionale Anforderungen}

Fachliche (funktionale) Anforderungen bestimmen die Korrektheit und Sicherheit der Software, aber auch teilweise die Benutzerfreundlichkeit.

\section{Nichtfunktionale Anforderungen}

Klassische technische (nicht-funktionale) Anforderungen sind:

Zuverlässigkeit (Robustheit)
Änderbarkeit (Wiederverendbarkeit)
Übertragbarkeit (Kompatibilität)
Effizienz (geringerer Ressourceneinsatz)

\section{Architektur}

Diagramm mit und grundlegende Designentscheidungen

\section{Komponenten}

Die Komponenten des Systems die im Architektur Modell verwendet werden erklären.

\section{Datenmodell}

UML

\section{Aktivitäten}

BPMN
